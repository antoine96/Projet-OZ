\section*{Architecture and design of our project}
First of all, our project is only one file who is a functor. In this functor, we use one port for each zombie and one port for the brave. In this way, we can easily send a message to an actor. There are a lot of other possibilities. For example, we can use one port by square of the map but it seems more instinctive to do like us.
\\

Here's the description of the utility of each function :
\\

\begin{itemize}
\item[\texttt{ZombiesNumber} :] creates a list of \texttt{NZombies}\footnote{\texttt{NZombies} is the number of zombies we want on the map at the beginning of the game} values. Each value is a number between 1 and \texttt{NbZeros} \footnote{\texttt{NbZeros} is the number of free squares on the map ; the number of squares where we can put a zombie}
\\

\item[\texttt{DelZombie} :] removes a zombie of the \texttt{ListZombie}. We use it when the brave kills a zombie or when a zombie becomes angry and explodes.
\\

\item[\texttt{CheckCase} :] is a function which returns a boolean. This function is usefull to see if a square contains a food or something else.
\\

\item[\texttt{UpdateList} :] is a very important function because without it, we can't remove an object of the map or move the brave.
\\

\item[\texttt{MaxWidth} :] with this function, we can create a suitable Canvas.
\\

\item[\texttt{ListZombie} :] this function returns a list with the position of each zombie.
\\

\item[\texttt{RemplirListe} :] we have now the initial list who represents the map.
\\

\item[\texttt{DrawBox} :] with this function, we can draw images on the map (with the help of the \texttt{MapList} created by \texttt{RemplirListe}).
\\

\item[\texttt{InitLayout} :] this function creates the initial map. We put \texttt{MapList} as argument. We only use this function at the creation of the map.
\\

\item[\texttt{BuildZombiePort} :] with this function, we can create a port for each zombie.
\\

\item[\texttt{ChooseDirection} :] sends a random direction at the stream placed at the position N of the tuple.
\\

\item[\texttt{UpdateListZombie} :] can update \texttt{ListZombie} when a zombie moves.
\\

\item[\texttt{NiceZombie} et \texttt{NiceBrave} :] draws an image in function of the direction (to have a nice orientation of the zombies and of the brave).
\\

\item[\texttt{ZombiesMove} :] It's the second most important function of our implementation. It creates the IA of the zombies and, after that, they can interact with the elements of the map.
\\

\item[\texttt{Game} :] It's the most important function because it controls the brave. Without this function, the brave can't interact with the elements of the map.
\\

\end{itemize}
To solve the problem of concurrency, we use ports and, each time, we send the update list as return value of the \texttt{Game} and the \texttt{ZombiesMove} functions. Without this, we can't know how zombies and brave progress on the map so it's really an important return value to synchronize the zombies with/and the brave. At the beginning, we tried to synchronize with threads (as we've seen with the ping pong example) but it's a bit difficult to pass the \texttt{MapList} between \textt{Game} and \texttt{ZombiesMove}.\\

Our project has a lot of advantages :
\begin{enumerate}
\item The number of zombies can be as big as we want.
\\

\item The map can have another form than a square.
\\

\item When a zombie tries 10 times to move but he can't because of something near him, he becomes angry and explodes.
\\

\item There're four images for zombies and four for the brave so we can know where they look at. It's important because if the brave has bullets, zombies only kills him if they're behind.
\end{enumerate}
